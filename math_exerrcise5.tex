\documentclass[UTF8]{ctexart} % 支持中文
\usepackage{amsmath, amsthm}  % 提供定理环境
\usepackage{amssymb}          % 数学符号支持
% 定义对应的文章的小分块
% 定义定理环境样式
\newtheorem{definition}{定义}[section] % 定义环境
\newtheorem{theorem}{引理}[section]    % 引理环境

\begin{document}

\section{驻点与费马引理}

% 驻点定义
\begin{definition}[驻点]
设函数 \( f(x) \) 在点 \( x_0 \) 的某邻域内有定义。如果满足
\[
f'(x_0) = 0,
\]
则称 \( x_0 \) 为函数 \( f(x) \) 的一个\textbf{驻点}(或临界点)。
\end{definition}

% 费马引理
\begin{theorem}[费马引理]
设函数 \( f(x) \) 满足:
\begin{enumerate}
    \item 在点 \( x_0 \) 的某去心邻域 \( \mathring{U}(x_0) \) 内可导;
    \item \( f(x) \) 在 \( x = x_0 \) 处取得极值(极大值或极小值)。
\end{enumerate}
则 \( x_0 \) 必为 \( f(x) \) 的驻点,即
\[
f'(x_0) = 0.
\]
\end{theorem}

\end{document}