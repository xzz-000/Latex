\documentclass[12pt, a4paper, openany]{ctexbook} % 使用book文档类,12pt字体,A4纸,章节从任意页开始(openany)
\usepackage{geometry} % 页面边距设置
\usepackage{graphicx} % 插入图片
\usepackage{amsmath, amssymb} % 数学公式和符号
\usepackage{fancyhdr} % 页眉页脚
\usepackage{hyperref} % 超链接和书签
\usepackage{caption} % 图表标题
\usepackage{subcaption} % 子图支持
\usepackage{biblatex} % 参考文献(新版推荐)
\addbibresource{refs.bib} % 参考文献数据库(需自行创建refs.bib文件)
\usepackage{float}
% 页面边距设置
\geometry{left=2.5cm, right=2.5cm, top=3cm, bottom=3cm}

% 页眉页脚样式
\pagestyle{fancy}
\fancyhf{}
\fancyhead[LE,RO]{\thepage} % 偶数页左页眉,奇数页右页眉显示页码
\fancyhead[RE]{\leftmark}   % 奇数页左页眉显示章节名
\fancyhead[LO]{\rightmark}  % 偶数页右页眉显示小节名

% 超链接样式(取消红色边框)
\hypersetup{
    colorlinks=true,
    linkcolor=blue,
    citecolor=green,
    urlcolor=magenta
}

% 自定义命令示例
\newcommand{\keyterm}[1]{\textbf{#1}} % 加粗关键词

% 书籍信息
\title{LaTeX 书籍编写示例}
\author{xiaox}
\date{\today}

\begin{document}

% 前言部分(罗马数字页码)
\frontmatter
\maketitle % 生成标题页
\chapter{序言} % 前言章节
这本书展示了如何使用 LaTeX 编写结构化的文档...

\tableofcontents % 生成目录


% 正文部分(阿拉伯数字页码)
\mainmatter
\chapter{引言}
\section{研究背景}
LaTeX 是排版学术文献的强大工具\cite{lamport1994latex}。例如,数学公式:
\begin{equation}
    E = mc^2 \label{eq:energy}
\end{equation}
公式 \ref{eq:energy} 是著名的质能方程。

\section{插入图片}
如图 \ref{fig:example} 所示:
\begin{figure}[htbp]
    \centering
    \includegraphics[width=0.5\textwidth]{images/example-image-logo.png} % 示例图片(需替换为实际图片)
    \caption{示例图片}
    \label{fig:example}
\end{figure}

\chapter{方法}
\section{实验设计}
数据如表 \ref{tab:data} 所示:
\begin{table}[htbp]
    \centering
    \caption{实验数据}
    \begin{tabular}{|c|c|c|}
        \hline
        样本 & 参数A & 参数B \\ \hline
        1    & 10    & 0.5   \\ \hline
        2    & 15    & 0.8   \\ \hline
    \end{tabular}
    \label{tab:data}
\end{table}

\section{子图示例}
如图 \ref{fig:subfigs} 所示:
\begin{figure}[htbp]
    \centering
    \begin{subfigure}[b]{0.4\textwidth}
        \includegraphics[width=\textwidth]{images/example-image-logo.png}
        \caption{子图1}
    \end{subfigure}
    \hfill
    \begin{subfigure}[b]{0.4\textwidth}
        \includegraphics[width=\textwidth]{images/example-image-logo.png}
        \caption{子图2}
    \end{subfigure}
    \caption{多子图示例}
    \label{fig:subfigs}
\end{figure}

% 附录部分
\appendix
\chapter{附录内容}
这里是附录,可以放置补充材料或代码。

% 参考文献(需配合biblatex或bibtex)
\printbibliography[title=参考文献]

% 后记(可选)
\backmatter
\chapter{后记}
感谢所有帮助过我的人!

\end{document}