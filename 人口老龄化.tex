\documentclass[UTF8]{ctexart}
\usepackage[a4paper]{geometry}
\geometry{left=2.5cm,right=2.5cm,top=2.5cm,bottom=2.5cm}
\usepackage{abstract}
\usepackage{titling}
\usepackage{setspace}
\setstretch{1.5}

\title{中国人口老龄化趋势的多维度成因分析}
\author{社会学研究小组}
\date{\today}

\begin{document}

\maketitle
\pagestyle{plain}
\begin{abstract}
本文基于人口统计学数据与社会政策变迁,系统分析中国人口老龄化的演进轨迹。研究表明:1980年代生育高峰与计划生育政策的叠加效应、新生代生育观念的转变、医疗水平提升带来的寿命延长构成三大核心动因。当前总和生育率已跌破1.3的警戒线,60岁以上人口占比超10\%的深度老龄化社会形态将持续强化<x-preset class="no-tts reference-tag disable-to-doc" data-index="2">2</x-preset><x-preset class="no-tts reference-tag disable-to-doc" data-index="3">3</x-preset>。
\end{abstract}

% 生成目录
\tableofcontents
\newpage

\section{人口老龄化的现状图谱}
中国于2000年正式迈入老龄化社会(60岁以上人口占比突破10\%),截至2020年:
\begin{enumerate}
  \item \textbf{规模特征}:老年人口达1.32亿,占全球老年人口20\%,居世界首位<x-preset class="no-tts reference-tag disable-to-doc" data-index="2">2</x-preset>
  \item \textbf{速度特征}:完成老龄化转型仅用25年(短于发达国家的50-100年)
  \item \textbf{结构特征}:上海、北京等城市65岁以上人口占比超14\%,呈现“未富先老”格局<x-preset class="no-tts reference-tag disable-to-doc" data-index="2">2</x-preset>
\end{enumerate}

\section{历史政策的关键影响}
\subsection{8090年代人口基数效应}
1982-1990年60岁以上人口从0.77亿增至近1亿,形成首批老龄化主力军。此阶段:
\begin{itemize}
  \item 人口中位年龄五年增长5\%<x-preset class="no-tts reference-tag disable-to-doc" data-index="2">2</x-preset>
  \item 老年抚养比从8.81\%跃升至20.14\%<x-preset class="no-tts reference-tag disable-to-doc" data-index="2">2</x-preset>
\end{itemize}

\subsection{计划生育的长期约束}
\begin{itemize}
  \item \textbf{政策强度}:1980-2015年生育限制导致新生人口断崖下跌<x-preset class="no-tts reference-tag disable-to-doc" data-index="3">3</x-preset>
  \item \textbf{代际失衡}:2021年推出三孩政策时,育龄妇女总量较2016年下降18\%<x-preset class="no-tts reference-tag disable-to-doc" data-index="3">3</x-preset>
  \item \textbf{结构扭曲}:少年人口占比持续萎缩,老年抚养负担倍增<x-preset class="no-tts reference-tag disable-to-doc" data-index="1">1</x-preset>
\end{itemize}

\section{新生代生育观的结构性转变}
\subsection{经济成本制约}
\begin{itemize}
  \item 城市化进程中教育成本增长300\%,住房成本增长580\%<x-preset class="no-tts reference-tag disable-to-doc" data-index="3">3</x-preset>
  \item 养育单个子女至18岁的成本达人均GDP的6.9倍
\end{itemize}

\subsection{价值观变迁}
\begin{itemize}
  \item 婚育年龄推迟:女性初婚年龄从1990年22岁延至2020年26.3岁
  \item 生育意愿衰减:育龄群体理想子女数降至1.8个,实际生育率仅1.3<x-preset class="no-tts reference-tag disable-to-doc" data-index="3">3</x-preset>
  \item 个体主义兴起:生育的“工具理性”弱化,丁克家庭比例突破12\%<x-preset class="no-tts reference-tag disable-to-doc" data-index="3">3</x-preset>
\end{itemize}

\section{多维驱动下的发展趋势}
\begin{enumerate}
  \item \textbf{加速深化期}(2025-2035):1962-1975年生育高峰人群进入老年
  \item \textbf{超级老龄化}(2040-):60岁以上人口占比将突破30\%
  \item \textbf{持续低生育陷阱}:即使生育政策放开,总和生育率难超1.5<x-preset class="no-tts reference-tag disable-to-doc" data-index="3">3</x-preset>
\end{enumerate}

\section*{结论}
中国老龄化是政策干预、经济转型与文化演进共同作用的结果。8090年代的人口积淀与计划生育形成历史基底,而医疗进步延长寿命叠加新生代生育意愿衰减,使老龄化进程呈现不可逆的加速态势。需构建包含养老金改革、银发产业培育、生育支持政策的综合治理体系。

\end{document}