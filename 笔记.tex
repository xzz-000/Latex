\documentclass[UTF8]{ctexart}
\usepackage{mathtools}
\title{tex日常笔记}
\author{xiaox}


\begin{document}


\maketitle
% 生成目录xelatex 笔记.tex xelatex 笔记.tex
\tableofcontents
\newpage
\pagestyle{plain}
%注意目录要两次才能生成,而且生成目录的命令必须在\pagestyle{plain}命令之前,且生成目录的命令前面最好要有\maketitle,让目录在作者标题等介绍的下面


\section{DeclarePairedDelimiter}

\textbackslash mathtools 宏包(amsmath 的扩展,已包含 \textbackslash text{\textbackslash textbackslash DeclarePairedDelimiter})。
\textbackslash text{\textbackslash textbackslash DeclarePairedDelimiter}是专门为定界符设计的,比手动用 \textbackslash newcommand 更安全(避免嵌套数学模式问题)。

命令名:自定义的新命令(如 \textbackslash paren)。
左符号 和 右符号:配对的定界符(如 ( )、[ ]、\textbackslash { \textbackslash })。
功能说明
自动调整符号大小
通过 \textbackslash  paren* 或 \textbackslash  paren[大小] 选项,符号会根据内容高度自动调整(类似 \textbackslash left( \textbackslash right) 的效果)。

三种调用方式

\textbackslash paren{内容}:默认大小(不自动调整)。
\textbackslash paren*{内容}:自动调整大小(等价于 \textbackslash left( 内容 \textbackslash right))。
\textbackslash paren[大小]{内容}:手动指定大小(如 \textbackslash big、\textbackslash Big、\textbackslash bigg 等)。

\section{三个数学宏包的介绍,区别和关系}
\begin{enumerate}
 \item amsmath	独立	多行公式、矩阵、基础排版	所有数学文档(必备)
  \item amssymb	独立(依赖 amsfonts)	扩展数学符号(如 \textbackslash mathbb)	需要特殊符号的文档
   \item mathtools	自动加载 amsmath	自动括号、高级排版工具	复杂数学公式(如论文、书籍)
\end{enumerate}

\section{ctex宏包}
 $\textbackslash textbackslash usepackage{ctex}$ 是 LaTeX 中用于支持中文排版的核心宏包,尤其适合处理中文、日文、韩文等东亚语言的文档。
它解决了原生 LaTeX 对中文支持不足的问题,提供了自动字体配置、标点压缩、中文断行等关键功能。
\textbackslash section{输出反斜杠}
(方式主要)输出这个:  \textbackslash textbackslash textbackslash
\section{生成目录}
注意目录(跟图片的引用一样 )要两次编译(例如:xelatex 笔记.tex xelatex 笔记.tex)才能生成,而且生成目录的命令必须在\pagestyle{plain}命令之前,
且生成目录的命令前面最好要有\maketitle,让目录在作者标题等介绍的下面
\section{行间公式和行内公式}
\begin{enumerate} 
 \item 行内公式:$ x = \frac{a}{b} + c $ → 小号分数  \\
 \item 行间公式:\[ x = \frac{a}{b} + c \] → 大号分数
 \item \textbackslash tfrac{ }{ }(文本样式分式命令):
       它强制分式始终以行内公式的分式大小显示(t 代表 text style)。
       无论它是在行内公式还是行间公式中使用,它都保持较小字体。
      使用场景:
      当在行间公式中需要保持紧凑格式时,避免某个分式显得过大破坏整体紧凑感。
\item 同理还有\textbackslash dfrac{}{}  这个是默认行间公式
      但是\textbackslash frac{ }{ }是它会根据当前所处的数学环境自动调整分式的大小。
在行内公式(用  括起来的公式)中,它会使用较小的字体显示,以保证行间距不至于过大。
在行间公式(用 括起来的公式,或 `equation` 等环境)中,它会使用较大的字体显示,以提高可读性。
\end{enumerate}
\section{题目的分值}
\subsection{第一方式}
用\pointpoints{分}{分} % 单数/复数形式均设为中文"分
然后仅需输入类似这样的程序:\question[17] 会输出 “17分”
\subsection{第二方式}
用\renewcommand{\pointformat}{[\arabic{points} 分]} % 格式如 [5分]
然后仅需输入类似这样的程序:
\begin{questions}
\question[17]
\end{questions} 
会输出 “17分”
\sucsection{注意}
无论是这两种方式中的哪一种,都需要在\textbackslash documentclass \texttt{ [UTF8]} \{exam\}
这一环境下执行
\end{document}