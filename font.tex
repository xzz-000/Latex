% !TEX program = xelatex
\documentclass[UTF8]{ctexart}
\setCJKmainfont{SimSun} % 显式指定中文字体(确保系统已安装)
\setCJKsansfont{SimHei} % 无衬线字体
\setCJKmonofont{FangSong} % 等宽字体

\begin{document}
1.0
默认字号(\texttt{\textbackslash normalsize}): 这是正文内容。

{\tiny 这是 \texttt{\textbackslash tiny} 字号。} \\
{\Large 这是 \texttt{\textbackslash Large} 字号。} \\
{\Huge 这是 \texttt{\textbackslash Huge} 字号。}

\vspace{2\baselineskip}
2.0
\title{\Huge 这是一个大标题}
\author{\Large 作者姓名}
\date{\small \today}
\maketitle

\section{章节标题} % 移除显式字号,由 ctexart 自动控制
正文内容使用 \texttt{\textbackslash normalsize}。

\subsection{子章节标题}
子章节标题默认与正文同字号,可通过命令调整。
正文内容。
{\footnotesize 这是脚注大小的注释文本。}

{\bfseries\Large 这是加粗的大号文本。} % 确保系统字体支持加粗

\vspace{2\baselineskip}
3.
\begin{center}
    \section*{字体尺寸命令演示} % 避免在标题中使用 \section* 嵌套
\end{center}

% 改用 \centering 避免多余间距
\begin{center}
    {\tiny 极小字号 (tiny)} \\
    {\scriptsize 较小字号 (scriptsize)} \\
    {\footnotesize 脚注大小 (footnotesize)} \\
    {\small 稍小字号 (small)} \\
    {\normalsize 默认字号 (normalsize)} \\
    {\large 稍大字号 (large)} \\
    {\Large 更大字号 (Large)} \\
    {\LARGE 极大字号 (LARGE)} \\
    {\huge 巨大字号 (huge)} \\
    {\Huge 最大字号 (Huge)}
\end{center}

\vspace{1em}
4.0
{\bfseries\Large 组合使用:加粗大号文本}

\vspace{1em}
\noindent 自定义字号: \\
{\fontsize{15}{18}\selectfont 15pt 字号,18pt 行距} \\
{\fontsize{25}{30}\selectfont 25pt 字号,30pt 行距}

\end{document}