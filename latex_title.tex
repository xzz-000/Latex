\documentclass[UTF8]{ctexart} % 使用 ctexart 文档类支持中文
\usepackage[titletoc]{appendix} % 可选:附录支持
\setcounter{tocdepth}{3} % 设置目录层次深度显示到 subsubsection(第三级)
\usepackage{graphicx} % 支持插入图片
\usepackage{float} % 增强浮动体控制

\begin{document}

% ===== 标题页 =====
\title{三层级目录结构示例}
\author{薛中州}
\date{\today}
\maketitle
\thispagestyle{empty} % 标题页无页脚(可选)

% ===== 目录 =====

\pagestyle{plain} % 确保正文页脚显示页码 \pagestyle{plain} 的作用
%功能:设置页眉(header)和页脚(footer)的样式。
%plain 样式:仅显示页码(默认在页脚中央),不显示章节标题或其他信息。
%其他常见样式:
%empty:不显示页眉和页脚(常用于标题页)。
%headings:显示当前章节标题和页码(页眉)。
\tableofcontents%自动生成文档的目录(基于 \section、\subsection 等标题)。

% ===== 正文层次结构 =====
\clearpage
\section{第一章 一级标题}
这是第一章的内容概述。

\subsection{第一节 二级标题}
这是第一节的内容概述。

\subsubsection{第一小节 三级标题}
这是第一小节的具体内容。

\subsubsection{第二小节 三级标题}
这是第二小节的具体内容。

\subsection{第二节 二级标题}
这是第二节的内容概述。

\clearpage
\section{第二章 一级标题}
这是第二章的内容概述。

\subsection{第一节 二级标题}
这是第一节的内容概述。

\subsubsection{第一小节 三级标题}
这是第一小节的具体内容。

\clearpage % 确保浮动体输出后分页
\section{第三章 一级标题}
第三章部分内容。

\end{document}